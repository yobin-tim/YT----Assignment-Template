\documentclass{article} % kind of document 
\usepackage[utf8]{inputenc} %encoding of choice
\usepackage[american]{babel} %language of choice
\usepackage{fancyhdr} %for header
\usepackage{amsmath} %math mode
\usepackage{amssymb} %math symbols
\usepackage{amsthm} %math theorem
\usepackage{enumerate} %make lists
\usepackage{graphicx} %insert images
\usepackage{float} %to fix image position
\usepackage{moreverb} %to make boxes
\usepackage{hyperref} %to create hyperlinks
\usepackage{lipsum} %lorem ipsum package
\usepackage{setspace} % to use singlespace below in the solution environment
\usepackage[us]{datetime} %package for setting due date in US format
\newdate{duedate}{05}{08}{2020} %to set a due date

\usepackage[margin=1in]{geometry}
\pagestyle{fancy}


\lhead{Due: \displaydate{duedate}}
\chead{Top Centre}
\rhead{Top Right}


\DeclareMathOperator*{\E}{\mathbb{E}} %ease of writing e and E
\newcommand{\e}{\mathrm{e}}

\newtheorem{theorem}{theorem} % Theorem display format
\newtheorem{problem}[theorem]{Problem} % Problem display format, last bracket sets display choice

\newenvironment{solution}[1][Answer]{\begin{singlespace}\underline{\textbf{#1:}}\quad }{\ \rule{0.3em}{0.3em}\end{singlespace}} % Answer format

\begin{document}

\begin{problem}
\normalfont
\textbf{(Main Question Header)}

\noindent \lipsum[1]
\end{problem}

\begin{solution}
We have, 
$$ \alpha  \beta  \gamma \Gamma \delta \Delta $$ % the double dollar signs create a Math env and centralize the content


\textbf{Note} that, when starting out, you can find several sources on the internet to help you find mathematical symbols that are \textit{predefined} in the TEX package.

The source that I use most is linked \hyperlink{https://oeis.org/wiki/List_of_LaTeX_mathematical_symbols}{here}.

\underline{A sidenote:} There are a few symbols and expressions that are either harder to find on the internet or have multiple ways to be achieved. I will list them below.

\begin{itemize}
    \item plims: $ \overset{p}{\longrightarrow} $
    
    \item Matrices
            $$V = \begin{bmatrix} 
    v_{11} & v_{12} &v_{13} \\v_{21} & v_{22} &v_{23} \\ v_{31} & v_{32} & v_{33}
    \end{bmatrix}$$
    There are other shapes of matrices that you can easily create
    $$ \begin{pmatrix} 
    v_{11} & v_{12} &v_{13} \\v_{21} & v_{22} &v_{23} \\ v_{31} & v_{32} & v_{33}
    \end{pmatrix} $$
    $$\begin{vmatrix} 
    v_{11} & v_{12} &v_{13} \\v_{21} & v_{22} &v_{23} \\ v_{31} & v_{32} & v_{33}
    \end{vmatrix}  $$
    
    \item Widehats: $\widehat{Wide}$ as opposed to $\hat{narrow}$
    
    \item Make brackets as big as the expressions inside using \textit{left} and \textit{right} commands:
    $$  = (\hat{\beta}_2 - \hat{\beta}_3) \left(\begin{bmatrix} 
    0 & 1 & -1
    \end{bmatrix} \frac{1}{n} \begin{bmatrix} 
    v_{11} & v_{12} &v_{13} \\v_{21} & v_{22} &v_{23} \\ v_{31} & v_{32} & v_{33}
    \end{bmatrix} \begin{bmatrix} 
    0 \\ 1 \\ -1
    \end{bmatrix}\right)^{-1} (\hat{\beta}_2 - \hat{\beta}_3) $$
    as opposed to 
    $$  = (\hat{\beta}_2 - \hat{\beta}_3) (\begin{bmatrix} 
    0 & 1 & -1
    \end{bmatrix} \frac{1}{n} \begin{bmatrix} 
    v_{11} & v_{12} &v_{13} \\v_{21} & v_{22} &v_{23} \\ v_{31} & v_{32} & v_{33}
    \end{bmatrix} \begin{bmatrix} 
    0 \\ 1 \\ -1
    \end{bmatrix})^{-1} (\hat{\beta}_2 - \hat{\beta}_3) $$
    
    \item Box your final answers to make them stand out using \textit{box}:
    $$ \boxed{\frac{\hat{\beta}_2}{\hat{\beta}_3^3} = 1 \text{ and } \hat{\beta}_2 = 1} $$
    
\end{itemize}

Also, if you don't leave a space in your \LaTeX{} code before ending the \textit{solution} environment, the small black square will appear beside the end of the answer instead of a line below.
\end{solution}

\begin{problem}
\normalfont
\textbf{(Main question header for the second questions)}

\noindent 
\begin{enumerate}[(i)]
    \item Subquestion
    \begin{solution}
    
    \end{solution}
    
    \item Another subquestion
    \begin{solution}
    This is just to show that the solution can be written inside the \textit{problem} environment.
    
    Also check out the additional (optional) argument taken by \textit{enumerate} that lets you choose between different kinds of numbered lists.
    
    \end{solution}
\end{enumerate}
\end{problem}


\end{document}
